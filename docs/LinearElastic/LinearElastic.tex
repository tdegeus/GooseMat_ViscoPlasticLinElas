%!TEX program = xelatex
\documentclass[times,namecite]{goose-article}

\title{%
  GooseSolid/LinearElastic
}
% Former GooseFEM mat1001

\author{T.W.J.~de~Geus}

\contact{%
  $^*$Contact: %
  \href{mailto:tom@geus.me}{tom@geus.me} %
  \hspace{1mm}--\hspace{1mm} %
  \href{http://www.geus.me}{www.geus.me}%
  \hspace{1mm}--\hspace{1mm} %
  \href{https://github.com/tdegeus/GooseSolid}{https://github.com/tdegeus/GooseSolid}%
}

\hypersetup{pdfauthor={T.W.J. de Geus}}

\header{%
  \href{https://github.com/tdegeus/GooseSolid}{GooseSolid/LinearElastic}%
}

\newcommand\leftstar[1]{\hspace*{-.3em}~^\star\!#1}

\begin{document}

\maketitle

\begin{abstract}
Linear elasticity: a linear relation between the Cauchy stress $\bm{\sigma}$ and the linear strain $\bm{\varepsilon}$

The model is implemented in 3-D, hence it can directly be used for either 3-D or 2-D plane strain problems.
\end{abstract}

\keywords{linear elasticity}

% \setcounter{tocdepth}{2}
% \tableofcontents

% \vfill\newpage
\section{Constitutive model}

The stress, $\bm{\sigma}$, is set by to the elastic strain, $\bm{\varepsilon}_\mathrm{e}$, through the following linear relation:
\begin{equation}
  \bm{\sigma} = \mathbb{C}_\mathrm{e} : \bm{\varepsilon}_\mathrm{e}
\end{equation}
wherein $\mathbb{C}_\mathrm{e}$ is the elastic stiffness, which reads:
\begin{align}\label{eq:model:elas}
  \mathbb{C}_\mathrm{e}
  &= K \bm{I} \otimes \bm{I} + 2 G (  \mathbb{I}_\mathrm{s} - \tfrac{1}{3} \bm{I} \otimes \bm{I} )
  \\
  &= K \bm{I} \otimes \bm{I} + 2 G \, \mathbb{I}_\mathrm{d}
\end{align}
with $K$ and $G$ the bulk and shear modulus respectively. See Appendix~\ref{sec:nomenclature} for nomenclature.

% \vfill\newpage
\appendix

\section{Nomenclature}
\label{sec:nomenclature}

\begin{itemize}
%
\item Dyadic tensor product
\begin{align}
  \mathbb{C} &= \bm{A} \otimes \bm{B} \\
  C_{ijkl}   &= A_{ij} \,      B_{kl}
\end{align}
%
\item Double tensor contraction
\begin{align}
  C &= \bm{A} : \bm{B} \\
    &= A_{ij} \, B_{ji}
\end{align}
%
\item Deviatoric projection tensor
%
\begin{equation}
  \mathbb{I}_\mathrm{d}
  = \mathbb{I}_\mathrm{s} - \tfrac{1}{3} \bm{I} \otimes \bm{I}
\end{equation}
%
\end{itemize}

% \bibliography{library}

\end{document}
