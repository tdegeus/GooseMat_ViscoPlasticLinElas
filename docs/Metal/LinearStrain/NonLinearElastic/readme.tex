%!TEX program = xelatex
\documentclass[times,namecite]{goose-article}

\title{%
  GooseMaterial/Metal/LinearStrain/NonLinearElastic
}
% Former GooseFEM mat1101

\author{T.W.J.~de~Geus}

\contact{%
  $^*$Contact: %
  \href{mailto:tom@geus.me}{tom@geus.me} %
  \hspace{1mm}--\hspace{1mm} %
  \href{http://www.geus.me}{www.geus.me}%
  \hspace{1mm}--\hspace{1mm} %
  \href{https://github.com/tdegeus/GooseMaterial}{https://github.com/tdegeus/GooseMaterial}%
}

\hypersetup{pdfauthor={T.W.J. de Geus}}

\header{%
  \href{https://github.com/tdegeus/GooseMaterial}{GooseMaterial/Metal/LinearStrain/NonLinearElastic} -- \href{http://www.geus.me}{T.W.J.\ de Geus}%
}

\newcommand\leftstar[1]{\hspace*{-.3em}~^\star\!#1}

\begin{document}

\maketitle

\begin{abstract}
This constitutive model encompasses a non-linear, but history independent, relation between the Cauchy stress, $\bm{\sigma}$, and the linear strain tensor, $\bm{\varepsilon}$, i.e.:
\begin{equation*}
  \bm{\sigma} = f \left( \bm{\varepsilon} \right)
\end{equation*}

The model is implemented in 3-D, hence it can directly be used for either 3-D or 2-D plane strain problems.
\end{abstract}

\keywords{non-linear elasticity}

\setcounter{tocdepth}{2}
\tableofcontents

\vfill\newpage
\section{Constitutive model}

The following strain-energy is defined:
%
\begin{equation}
  W ( \bm{\varepsilon} )
  = \frac{9}{2} K \varepsilon_\mathrm{m}^2
  + \frac{ \sigma_0 \, \varepsilon_0 }{ n+1 }
    \left( \frac{\varepsilon_\mathrm{eq}}{\mathrm{\varepsilon_0}} \right)^{n+1}
\end{equation}
%
where $K$ is the bulk modulus, $\varepsilon_0$ and $\sigma_0$ are a reference strain and stress respectively, and $n$ is an exponent that sets the degree of non-linearity. Finally $\varepsilon_\mathrm{m}$ and $\varepsilon_\mathrm{eq}$ are the hydrostatic and equivalent strains (see Appendix~\ref{sec:strain}).

This leads to the following stress--strain relation:
%
\begin{equation}
\label{eq:stress}
  \bm{\sigma}
  = \frac{\partial W}{\partial \bm{\varepsilon}}
  = 3 K \varepsilon_\mathrm{m} \, \bm{I}
  + \frac{2}{3} \frac{\sigma_0}{\varepsilon_0^n} \,
    \varepsilon_\mathrm{eq}^{n-1} \, \bm{\varepsilon}_\mathrm{d}
\end{equation}
%
see Appendix~\ref{sec:nomenclature} for nomenclature.

\section{Consistent tangent}

The consistent tangent maps a variation in strain, $\delta \bm{\varepsilon}$, to a variation in stress, $\delta \bm{\sigma}$, as follows
%
\begin{equation}
  \delta \bm{\sigma} = \mathbb{C} : \delta \bm{\varepsilon}
\end{equation}
%
The tangent, $\mathbb{C}$, thus corresponds to the derivative of \eqref{eq:stress} w.r.t.\ strain. For this, the chain rule is employed:
%
\begin{equation}
  \mathbb{C}
  = \frac{\partial}{\partial \bm{\varepsilon}}
    \bigg[\;
      3 K \varepsilon_\mathrm{m} \, \bm{I}
    \;\bigg]
  + \frac{\partial}{\partial \bm{\varepsilon}_\mathrm{d}}
    \bigg[\;
      \frac{2}{3} \frac{\sigma_0}{\varepsilon_0^n} \,
      \varepsilon_\mathrm{eq}^{n-1} \, \bm{\varepsilon}_\mathrm{d}
    \;\bigg]
  : \frac{\partial \bm{\varepsilon}_\mathrm{d}}{\partial \bm{\varepsilon}}
\end{equation}
%
Where:
\begin{itemize}
%
\item the derivative of the volumetric part reads
%
\begin{equation}
  \frac{\partial}{\partial \bm{\varepsilon}}
  \bigg[\;
    3 K \varepsilon_\mathrm{m} \, \bm{I}
  \;\bigg]
  = K \bm{I} \otimes \bm{I}
\end{equation}
%
\item the chain rule for the deviatoric part reads
%
\begin{align}
  \frac{\partial}{\partial \bm{\varepsilon}_\mathrm{d}}
  \bigg[\;
    \varepsilon_\mathrm{eq}^{n-1} \, \bm{\varepsilon}_\mathrm{d}
  \;\bigg]
  &=
  \frac{
    \partial \big[ \varepsilon_\mathrm{eq}^{n-1} \big]
  }{
    \partial \bm{\varepsilon}_\mathrm{d}
  } \otimes \bm{\varepsilon}_\mathrm{d}
  + \varepsilon_\mathrm{eq}^{n-1}
  \frac{
    \partial \bm{\varepsilon}_\mathrm{d}
  }{
    \partial \bm{\varepsilon}_\mathrm{d}
  }
  \\
  &=
  \tfrac{2}{3} (n-1) \, \varepsilon_\mathrm{eq}^{n-3} \,
  \bm{\varepsilon}_\mathrm{d} \otimes \bm{\varepsilon}_\mathrm{d}
  + \varepsilon_\mathrm{eq}^{n-1} \, \mathbb{I}
\end{align}
%
\item and it has been used that
%
\begin{align}
  \frac{\partial}{\partial \bm{\varepsilon}_\mathrm{d}}
  \bigg[\;
    \varepsilon_\mathrm{eq}^{n-1}
  \;\bigg]
  &= (n-1)\, \varepsilon_\mathrm{eq}^{n-2} \,
  \frac{2}{3} \frac{\bm{\varepsilon}_\mathrm{d}}{\varepsilon_\mathrm{eq}}
  \\
  &= \tfrac{2}{3} (n-1) \,
    \varepsilon_\mathrm{eq}^{n-3} \, \bm{\varepsilon}_\mathrm{d}
\end{align}
%
\end{itemize}
%

Combining the above yields:
\begin{align}
\mathbb{C}
&= K \bm{I} \otimes \bm{I} +
\frac{2}{3} \frac{\sigma_0}{\varepsilon_0^n}
\bigg(
  \tfrac{2}{3} (n-1) \, \varepsilon_\mathrm{eq}^{n-3}
  \bm{\varepsilon}_\mathrm{d} \otimes \bm{\varepsilon}_\mathrm{d}
  + \varepsilon_\mathrm{eq}^{n-1} \mathbb{I}
\bigg) : \mathbb{I}_\mathrm{d} \\
&= K \bm{I} \otimes \bm{I} +
\frac{2}{3} \frac{\sigma_0}{\varepsilon_0^n}
\bigg(
  \tfrac{2}{3} (n-1) \, \varepsilon_\mathrm{eq}^{n-3}
  \bm{\varepsilon}_\mathrm{d} \otimes \bm{\varepsilon}_\mathrm{d}
  + \varepsilon_\mathrm{eq}^{n-1} \, \mathbb{I}_\mathrm{d}
\bigg)
\end{align}

\appendix
\vfill\newpage

\section{Nomenclature}
\label{sec:nomenclature}

\begin{itemize}
%
\item Dyadic tensor product
\begin{align}
  \mathbb{C} &= \bm{A} \otimes \bm{B} \\
  C_{ijkl}   &= A_{ij} \,      B_{kl}
\end{align}
%
\item Double tensor contraction
\begin{align}
  C &= \bm{A} : \bm{B} \\
    &= A_{ij} \, B_{ji}
\end{align}
%
\item Deviatoric projection tensor
\begin{equation}
  \mathbb{I}_\mathrm{d}
  = \mathbb{I}_\mathrm{s} - \tfrac{1}{3} \bm{I} \otimes \bm{I}
\end{equation}
%
\end{itemize}

\section{Strain measures}
\label{sec:strain}

\begin{itemize}
%
\item Mean strain
\begin{equation}
  \varepsilon_\mathrm{m}
  = \tfrac{1}{3} \, \mathrm{tr} ( \bm{\varepsilon} )
  = \tfrac{1}{3} \, \bm{\varepsilon} : \bm{I}
\end{equation}
%
\item Strain deviator
\begin{equation}
  \bm{\varepsilon}_\mathrm{d}
  = \bm{\varepsilon} - \varepsilon_\mathrm{m} \, \bm{I}
  = \mathbb{I}_\mathrm{d} : \bm{\varepsilon}
\end{equation}
%
\item Equivalent strain
\begin{equation}
  \varepsilon_\mathrm{eq}
  = \; \sqrt{
    \tfrac{2}{3} \, \bm{\varepsilon}_\mathrm{d} : \bm{\varepsilon}_\mathrm{d}
  }
\end{equation}
%
\end{itemize}

\section{Variations}
\label{sec:variations}

\begin{itemize}
%
\item Strain deviator
\begin{equation}
  \delta \bm{\varepsilon}_\mathrm{d}
  = \left( \mathbb{I}_\mathrm{s} - \tfrac{1}{3} \bm{I} \otimes \bm{I} \right) :
    \delta \bm{\varepsilon}
  = \mathbb{I}_\mathrm{d} : \delta \bm{\varepsilon}
\end{equation}
%
\item Mean equivalent strain
\begin{equation}
  \delta \varepsilon_\mathrm{m}
  = \tfrac{1}{3} \bm{I} : \delta \bm{\varepsilon}
\end{equation}
%
\item Von Mises equivalent strain
\begin{align}
  \delta \varepsilon_\mathrm{eq}
  &= \frac{1}{3} \frac{1}{\varepsilon_\mathrm{eq}}
     \left( \bm{\varepsilon}_\mathrm{d} : \delta \bm{\varepsilon}_\mathrm{d} +
     \delta \bm{\varepsilon}_\mathrm{d} : \bm{\varepsilon}_\mathrm{d} \right) \\
  &= \frac{2}{3} \frac{1}{\varepsilon_\mathrm{eq}}
     \left( \bm{\varepsilon}_\mathrm{d} : \delta \bm{\varepsilon}_\mathrm{d} \right) \\
  &= \frac{2}{3} \frac{\bm{\varepsilon}_\mathrm{d}}{\varepsilon_\mathrm{eq}} :
     \delta \bm{\varepsilon}_\mathrm{d}
\end{align}
%
\end{itemize}


% \bibliography{library}

\end{document}
