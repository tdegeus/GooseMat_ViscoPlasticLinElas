%!TEX program = xelatex
\documentclass[times,namecite]{goose-article}

\title{%
  GooseMaterial/AmorphousSolid/LinearStrain/ElasticLiquid
}

\author{T.W.J.~de~Geus}

\contact{%
  $^*$Contact: %
  \href{mailto:tom@geus.me}{tom@geus.me} %
  \hspace{1mm}--\hspace{1mm} %
  \href{http://www.geus.me}{www.geus.me}%
  \hspace{1mm}--\hspace{1mm} %
  \href{https://github.com/tdegeus/GooseMaterial}{https://github.com/tdegeus/GooseMaterial}%
}

\hypersetup{pdfauthor={T.W.J. de Geus}}

\header{%
  \href{https://github.com/tdegeus/GooseMaterial}{GooseMaterial/AmorphousSolid/LinearStrain/ElasticLiquid -- \href{http://www.geus.me}{T.W.J.\ de Geus}}%
}

\newcommand\leftstar[1]{\hspace*{-.3em}~^\star\!#1}

% Former GooseFEM mat2002

\begin{document}

\maketitle

\begin{abstract}
This materials responds linear elastically up to a certain threshold stress. Then it behaves like a Newtonian fluid for a set period of time.

The model is implemented in 3-D, hence it can directly be used for either 3-D or 2-D plane strain problems.
\end{abstract}

\keywords{elasto-plasticity; linear elasticity}

\setcounter{tocdepth}{2}
\tableofcontents

\vfill\newpage
\section{Constitutive model}

The model is formulated in rate form:
\begin{equation}
\begin{cases}
\dot{\sigma}_\mathrm{m} = 3 K \dot{\varepsilon}_\mathrm{m} \\
\dot{\bm{\sigma}}_\mathrm{d} = 2 G \dot{\bm{\varepsilon}}_\mathrm{d} - \eta \, \bm{\sigma}_\mathrm{d}
\end{cases}
\end{equation}
where $\dot{\sigma}_\mathrm{m}$ and $\dot{\bm{\sigma}}_\mathrm{d}$ the hydrostatic and deviator stress rate, $\dot{\varepsilon}_\mathrm{m}$ and $\dot{\bm{\varepsilon}}_\mathrm{d}$ the corresponding strain measures\footnote{The strain tensor $\bm{\varepsilon} = \tfrac{1}{2} \left( \vec{\nabla} \vec{u} + (\vec{\nabla} \vec{u})^T \right)$. For its rate the displacement $\vec{u}$ has to be replaced by its rate $\vec{v} = \dot{\vec{u}}$}, and $\eta$ the damping rate. See Appendix~\ref{sec:ap:nomenclature} for nomenclature, and Appendix~\ref{sec:ap:stress} for the precise definitions of the hydrostatic and deviatoric part of the stress tensor (these definitions are identical for the stress rate, the strain and its rate).

The model now has two stages:
\begin{itemize}
  \item An elastic stage, for which $\eta = 0$. This stage corresponds exactly to linear elasticity, i.e.
  \begin{equation}
    \bm{\sigma} = \mathbb{C} : \bm{\varepsilon}
  \end{equation}
  with
  \begin{equation}
    \mathbb{C}_\mathrm{e} = K \bm{I} \otimes \bm{I} + 2 G \, \mathbb{I}_\mathrm{d}
  \end{equation}
  %
  \item A fluid phase, in which $\eta \neq 0$. In this stage the stress relaxes exponentially to a value that is determined by the ratio of the loading rate and the damping rate (and the yield stress over the shear modulus).
\end{itemize}
These stage are controlled using two parameters:
\begin{enumerate}
  \item $\eta = 0 \rightarrow \eta \neq 0$ once the equivalent stress $\sigma_\mathrm{eq} = \sigma_\mathrm{y0}$.
  \item The material then relaxes for a certain time $T_\mathrm{fluid}$, after which $\eta \neq 0 \rightarrow \eta = 0$.
\end{enumerate}

\section{Implementation}

For the numerical implementation an implicit time integration scheme is selected. I.e.
\begin{equation}
  \bm{\sigma}^{(t+\Delta t)} = \bm{\sigma}^{(t)} + \Delta t \; \dot{\bm{\sigma}}^{(t+\Delta t)}
\end{equation}
The hydrostatic part of the stress can directly be evaluated
\begin{equation}
  \sigma_\mathrm{m}^{(t+\Delta t)} = \sigma_\mathrm{m}^{(t)} + \Delta t \, 3 K \, \dot{\varepsilon}_\mathrm{m}^{(t + \Delta t)}
\end{equation}
The deviator part reads
\begin{equation}
  \bm{\sigma}_\mathrm{d}^{(t+\Delta t)} = \bm{\sigma}_\mathrm{d}^{(t)} + \Delta t \, \left[ 2 G \, \dot{\bm{\varepsilon}}_\mathrm{d}^{(t+\Delta t)} - \eta \, \bm{\sigma}_\mathrm{d}^{(t+\Delta t)} \right]
\end{equation}
or
\begin{equation}
  \bm{\sigma}_\mathrm{d}^{(t+\Delta t)} = \frac{1}{1 + \eta \, \Delta t} \left[ \bm{\sigma}_\mathrm{d}^{(t)} + \Delta t \, 2 G \, \dot{\bm{\varepsilon}}_\mathrm{d}^{(t+\Delta t)} \right]
\end{equation}

\appendix
\vfill\newpage

\section{Nomenclature}
\label{sec:ap:nomenclature}

\begin{itemize}
%
\item Dyadic tensor product
\begin{align}
  \mathbb{C} &= \bm{A} \otimes \bm{B} \\
  C_{ijkl}   &= A_{ij} \,      B_{kl}
\end{align}
%
\item Double tensor contraction
\begin{align}
  C &= \bm{A} : \bm{B} \\
    &= A_{ij} \, B_{ji}
\end{align}
%
\item Deviatoric projection tensor
%
\begin{equation}
  \mathbb{I}_\mathrm{d}
  = \mathbb{I}_\mathrm{s} - \tfrac{1}{3} \bm{I} \otimes \bm{I}
\end{equation}
%
\end{itemize}


\section{Stress measures}
\label{sec:ap:stress}


\begin{itemize}
%
\item Mean stress
%
\begin{equation}
\sigma_\mathrm{m}
= \tfrac{1}{3} \, \mathrm{tr} ( \bm{\sigma} )
= \tfrac{1}{3} \, \bm{\sigma} : \bm{I}
\end{equation}
%
\item Stress deviator
%
\begin{equation}
  \bm{\sigma}_\mathrm{d}
  = \bm{\sigma} - \sigma_\mathrm{m} \, \bm{I}
  = \mathbb{I}_\mathrm{d} : \bm{\sigma}
\end{equation}
%
\item Von Mises equivalent stress
\begin{align}
\sigma_\mathrm{eq}
= \sqrt{ \tfrac{3}{2} \, \bm{\sigma}_\mathrm{d} : \bm{\sigma}_\mathrm{d} }
= \sqrt{ 3 J_2(\bm{\sigma}) }
\end{align}
where the second-stress invariant
\begin{align}
J_2 = \tfrac{1}{2} \, || \, \bm{\sigma}_\mathrm{d} \, ||^2
    = \tfrac{1}{2} \, \bm{\sigma}_\mathrm{d} : \bm{\sigma}_\mathrm{d}
\end{align}
%
\end{itemize}

\bibliography{library}

\end{document}
